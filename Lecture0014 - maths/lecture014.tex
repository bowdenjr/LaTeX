\documentclass{article}
\title{Mathematical Symbols}
\author{Jonathan Bowden}

\usepackage[margin=2cm]{geometry}
\usepackage[UKenglish]{isodate}

\begin{document}
\maketitle

\section{Introduction}

This document will create some mathematics.

\subsection{Formulae}

Mathematical equations can be created inline as $5*4=20$ and can include a number of different math symbols, such as $7 \leq  10$ is true.


You can also put your equation on a separate line as \[5^2 = 25\] to call out the formula.

\subsection{Float: Equations}

My first result is shown in Equation \ref{eq:minutes}, which shows the number of minutes in a day.

\begin{equation}
	24*60 = 1440
	\label{eq:minutes}
\end{equation}

There are a lot of other interesting mathematics sybols that I can learn, as shown in Equation \ref{eq:symbols}.

\begin{equation}
	7^4+6*17\geq 13 X \alpha 13 \exists \rightarrow Nonesense
	\label{eq:symbols}
\end{equation}

Here is a real equation for Person's Product Momement Correlation Coefficient for a sample:

\begin{equation}
	r_{xy} = \frac{n\sum{x_i y_i}- \sum{x_i} \sum{y_i}}
		{\sqrt{n\sum{x_i}^2 - (\sum{x_i})^2}\sqrt{n\sum{{y_i}^2 - (\sum{y_i})^2}}}
\end{equation}

\section{Conclusion}

This was a brief primer on \LaTeX mathematics.

\end{document}