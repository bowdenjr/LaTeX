\documentclass{article}
\title{Tables, Floats and the Tabular Environment}
\author{Jonathan Bowden}

\usepackage[margin=2cm]{geometry}
\usepackage[UKenglish]{isodate}

\begin{document}
\maketitle

\section{Introduction}
Learning to make some tables

\subsection{The Tabular Environment}

This is the tabular environment for the table shown in the Table ~\ref{tab:grades}.

\noindent There are 3 ways of creating space between paragraphs and tables.
\vspace{0.2cm} %vertical space
\noindent That was vspace.
\smallskip
\noindent That was small skip, the next is big skip.
\bigskip


\begin{tabular}{|c||c|}  %ll means left aligned, cc centred, use | for vertical bars, \hline for horiztonal bars
	\hline
	Name& Jonathan Bowden\\
	\hline
	Subject& Writing with \LaTeX \\
	\hline
	Grade& A* \\
	\hline
\end{tabular}

\subsection{Floats: Table}

\begin{table}[htbp] %h is here, LaTeX tries to put it in a sensible place, t = top of next page, b = bottom of next page, p = separate page
	\caption{My Gradebook}	
	\begin{center}
		\begin{tabular}{|c||c|}
			\hline
			Name& Jonathan Bowden\\
			\hline
			Subject& Writing with \LaTeX \\
			\hline
			Grade& A* \\
			\hline
		\end{tabular}
	\end{center}
	\label{tab:grades}
\end{table}

\subsection{References}

You can see my grades in the Table \ref{tab:grades}.  % ~ ensures the ref is put on the same line

\section{Conclusion}

Now I know how to create basic tables.

\end{document}